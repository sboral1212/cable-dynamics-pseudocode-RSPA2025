\documentclass[11pt]{article}
\usepackage[a4paper,margin=1in]{geometry}
\usepackage{amsmath,amssymb}
\usepackage{algorithm}
\usepackage{algorithmic}
\usepackage{booktabs}
\usepackage{tikz}
\usetikzlibrary{arrows.meta, positioning}
\usepackage{hyperref}

\title{\vspace{-1cm}
\textbf{Electronic Supplementary Material (ESM S1)}\\[4pt]
\large Pseudocode and Workflow for Periodic Cable Dynamics Solver}
\date{}
\author{}

\begin{document}
\maketitle

\section*{Overview and Code Availability}

The numerical simulations reported in the article
\emph{``Modeling and Simulation of Periodic Cable Dynamics with Bending Stiffness Using Time-Domain and Multi-Harmonic Balance Methods''}
rely on a MATLAB implementation that forms part of a larger, proprietary research
codebase that is actively being developed for multiple future publications.
For this reason, the full executable source code cannot be released publicly
in its current form.

To ensure transparency and reproducibility, this document provides a complete
and self-contained \textbf{non-executable pseudocode description} of the computational
workflow, including:
\begin{itemize}
  \item static configuration solver;
  \item time-domain dynamic solver;
  \item hydrodynamic force evaluation;
  \item post-processing (shape reconstruction and fairlead tension).
\end{itemize}
The pseudocode and flow diagram below fully describe the numerical procedure
and are sufficient for independent re-implementation of the methods presented
in the paper.

While the full MATLAB implementation is not publicly released, it may be shared
with bona fide researchers upon reasonable request for academic, non-commercial
use, subject to institutional and project constraints.

\section*{Model Parameters}

Table~\ref{tab:params} summarizes the main physical and numerical parameters.

\begin{table}[h!]
\centering
\begin{tabular}{@{}lll@{}}
\toprule
Symbol & Description & Typical value \\ \midrule
$N$       & Number of nodes along cable & $51$ \\
$L$       & Unstretched cable length    & $902.2\,\mathrm{m}$ \\
$EA$      & Axial stiffness             & $3.84243\times10^7\,\mathrm{N}$ \\
$EI$      & Bending stiffness           & (given in main paper) \\
$W$       & Submerged weight/length     & $698.094\,\mathrm{N/m}$ \\
$\rho_w$  & Water density               & $1025\,\mathrm{kg/m^3}$ \\
$d$       & Cable diameter              & $0.09\,\mathrm{m}$ \\
$C_d^t$   & Tangential drag coefficient & $1.6$ \\
$C_d^n$   & Normal drag coefficient     & $1.6$ \\
$m$       & Structural mass/length      & $77.7066\,\mathrm{kg/m}$ \\
$C_a$     & Added-mass coefficient      & $1.0$ \\
$\omega$  & Angular forcing frequency   & $2\pi f$ (see paper) \\
$A_x,A_z$ & Fairlead motion amplitudes  & (given in main paper) \\
$U_c,V_c$ & Current components          & (given in main paper) \\ \bottomrule
\end{tabular}
\caption{Model parameters used in the pseudocode description.}
\label{tab:params}
\end{table}

\newpage

\section*{Algorithm S1: Static Configuration (Pseudocode)}

\begin{algorithm}[h!]
\caption{Static configuration of the cable (non-executable pseudocode)}
\begin{algorithmic}[1]
\STATE Discretize arc length:
\[
s_i = \frac{i-1}{N-1}L,\quad i=1,\dots,N.
\]
\STATE Initialize arrays:
\begin{itemize}
  \item $\phi_{\text{s}}[i]$ \hfill (initial angle guess, e.g.\ catenary);
  \item $e_{\text{s}}[i]$ \hfill (initial axial strain guess).
\end{itemize}
\STATE For each interior node $i=2,\dots,N-1$:
\begin{itemize}
  \item approximate derivatives of $\phi_{\text{s}}$ w.r.t.\ $s$ by finite differences;
  \item compute curvature and curvature derivatives;
  \item set $T[i] = EA\,e_{\text{s}}[i]$ and approximate $dT/ds$ by central differences;
  \item form vertical and horizontal equilibrium residuals,
  representing bending, tension and weight balance at node $i$.
\end{itemize}
\STATE Impose boundary conditions by reconstructing geometry from
$(1+e_{\text{s}})[\cos\phi_{\text{s}},\sin\phi_{\text{s}}]$ so that
\[
(x,z)(0) = (0,0),\quad (x,z)(L) = (x_{\text{end}},z_{\text{end}}).
\]
\STATE Solve the nonlinear system for $\phi_{\text{s}}$ and $e_{\text{s}}$
using an iterative method (e.g.\ Newton-type) until all residuals and
boundary errors fall below a prescribed tolerance.
\STATE Return static fields $\phi_{\text{s}}[i]$ and $e_{\text{s}}[i]$.
\end{algorithmic}
\end{algorithm}

\section*{Algorithm S2: Time-Domain Dynamic Solver (Pseudocode)}

\begin{algorithm}[h!]
\caption{Time-domain simulation (non-executable pseudocode)}
\begin{algorithmic}[1]
\STATE Define dynamic state variables at node $i$:
\[
u[i](t),\; v[i](t),\; e[i](t),\; \phi[i](t).
\]
\STATE Initialize at $t=0$:
\[
u[i]=0,\quad v[i]=0,\quad e[i]=0,\quad \phi[i]=0\quad \forall i.
\]
\STATE For a given time $t$:
\begin{enumerate}
\item Compute fairlead velocity in global coordinates from the prescribed motion:
\[
\dot X_f(t)=A_x\,\omega(-\sin\omega t),\quad
\dot Z_f(t)=A_z\,\omega(\cos\omega t).
\]
\item Project $(\dot X_f,\dot Z_f)$ onto the local tangent/normal at node $N$
using the static angle plus dynamic correction; set $u[N]$, $v[N]$ accordingly.
\item Enforce anchored-end boundary conditions:
\[
u[1]=0,\quad v[1]=0.
\]
\end{enumerate}
\STATE For each interior node $i=2,\dots,N-1$:
\begin{enumerate}
\item Approximate spatial derivatives $du/ds$, $dv/ds$, $de/ds$, $d\phi/ds$ using
finite differences.
\item Define relative fluid velocities using the static orientation:
\[
u_{\text{rel}} = u - (U_c\cos\phi_{\text{s}}+V_c\sin\phi_{\text{s}}),
\]
\[
v_{\text{rel}} = v - (-U_c\sin\phi_{\text{s}}+V_c\cos\phi_{\text{s}}).
\]
\item Compute drag forces:
\[
F_t = -\tfrac12 \rho_w d C_d^t |u_{\text{rel}}|u_{\text{rel}}\sqrt{1+e},\quad
F_n = -\tfrac12 \rho_w d C_d^n |v_{\text{rel}}|v_{\text{rel}}\sqrt{1+e}.
\]
\item Assemble semi-discrete equations for each node:
\begin{align*}
\frac{du}{dt}   &= \frac{1}{m}\big[\text{bending terms} +
EA\,\frac{de}{ds} - W\,\phi\cos\phi_{\text{s}} + F_t\big],\\
\frac{dv}{dt}   &= \frac{1}{m+m_a}\big[\text{bending terms} +
T_{\text{s}}\frac{d\phi}{ds} + EA\,e\,\frac{d\phi_{\text{s}}}{ds}
+ W\,\phi\sin\phi_{\text{s}} + F_n\big],\\
\frac{de}{dt}   &= \frac{du}{ds} - v\,\frac{d\phi_{\text{s}}}{ds},\\
\frac{d\phi}{dt}&= \frac{dv}{ds} + u\,\frac{d\phi_{\text{s}}}{ds},
\end{align*}
where $T_{\text{s}}$ is the static tension field and ``bending terms''
collect contributions proportional to $EI$ and higher spatial derivatives
of $\phi$.
\end{enumerate}
\STATE Advance the state $y(t)=\{u,v,e,\phi\}$ from $t$ to $t+\Delta t$ using a
standard ODE time integrator (e.g.\ explicit Runge--Kutta). Repeat until
$t$ reaches $t_{\text{final}}$.
\end{algorithmic}
\end{algorithm}

\section*{Algorithm S3: Post-Processing (Pseudocode)}

\begin{algorithm}[h!]
\caption{Shape reconstruction and fairlead tension (non-executable pseudocode)}
\begin{algorithmic}[1]
\STATE At time $t_k$, compute total angle and strain:
\[
\phi_{\text{tot}}[i]=\phi_{\text{s}}[i]+\phi[i](t_k),\quad
e_{\text{tot}}[i]=e_{\text{s}}[i]+e[i](t_k).
\]
\STATE Reconstruct cable shape:
\[
x[1]=0,\quad z[1]=0,
\]
\[
x[i+1]=x[i] + ds\,(1+e_{\text{tot}}[i])\cos\phi_{\text{tot}}[i],
\]
\[
z[i+1]=z[i] + ds\,(1+e_{\text{tot}}[i])\sin\phi_{\text{tot}}[i],
\]
for $i=1,\dots,N-1$.
\STATE Compute fairlead tension:
\[
T_f(t_k)=T_{\text{s}}[N] + EA\,e[N](t_k),
\]
where $T_{\text{s}}[N]$ is the static tension at the fairlead.
\STATE Use $T_f(t)$ and the reconstructed shape $(x_i(t),z_i(t))$ for plotting
and comparison with the results shown in the main article.
\end{algorithmic}
\end{algorithm}

\newpage
\section*{Flow Diagram of the Workflow}

\begin{figure}[h!]
\centering
\small
\begin{tikzpicture}[
    node distance=0.65cm,
    block/.style={rectangle, draw, rounded corners, thick,
                  text width=5cm, align=center, minimum height=0.7cm,
                  inner sep=2pt},
    term/.style={ellipse, draw, thick,
                 minimum width=1.8cm, minimum height=0.6cm},
    decision/.style={diamond, draw, thick, aspect=1.6,
                     inner sep=1pt, align=center, text width=1.7cm},
    line/.style={-Stealth, thick}
]

\node[term] (start) {Start};

\node[block, below=of start] (input)
    {Input physical \& numerical parameters};

\node[block, below=of input] (static)
    {Static configuration:\\ solve for $\phi_{\text{s}}(s)$, $e_{\text{s}}(s)$};

\node[block, below=of static] (init)
    {Initialize dynamic state at $t=0$};

\node[block, below=of init] (bc)
    {Apply BCs at anchor and fairlead};

\node[block, below=of bc] (forces)
    {Compute relative velocities and hydrodynamic forces};

\node[block, below=of forces] (rhs)
    {Assemble semi-discrete ODE system};

\node[block, below=of rhs] (step)
    {Advance state one time step};

\node[decision, below=of step] (loop)
    {$t < t_f$ ?};

\node[block, below=of loop, yshift=-0.2cm] (post)
    {Reconstruct shape \& compute fairlead tension};

\node[term, below=of post] (end) {End};

\draw[line] (start) -- (input);
\draw[line] (input) -- (static);
\draw[line] (static) -- (init);
\draw[line] (init) -- (bc);
\draw[line] (bc) -- (forces);
\draw[line] (forces) -- (rhs);
\draw[line] (rhs) -- (step);
\draw[line] (step) -- (loop);
\draw[line] (loop) -- node[right]{No} (post);
\draw[line] (post) -- (end);

\draw[line] (loop.west) -- ++(-1.5,0) -- ++(0,4.4) -- (bc.west)
    node[midway,left]{Yes};

\caption{Compact flow diagram of the numerical procedure.}
\end{tikzpicture}
\end{figure}

\section*{Notes}

\begin{itemize}
  \item The pseudocode above is intentionally high-level and omits specific
        finite-difference stencils, solver tolerances and implementation details.
  \item This document is non-executable and contains no MATLAB or other runnable
        source code.
  \item The full MATLAB implementation used in the article is proprietary and
        may be shared, upon reasonable request, with bona fide researchers for
        academic, non-commercial purposes.
\end{itemize}

\end{document}
